\documentclass{beamer}
\usepackage{etex}
\usepackage[utf8]{inputenc}
\usepackage[T1]{fontenc}
\usepackage{amsmath}
\usepackage{amssymb}
\usepackage{stmaryrd}
\usepackage{bussproofs} 
\usepackage{tikz}
\usepackage{tikz-qtree} 
\usepackage{alltt}
\usepackage{listings}          % format code
% First define common settings for all languages
\usepackage{xcolor}
\definecolor{darkgreen}{rgb}{0,0.55,0.08} 
\definecolor{darkblue}{rgb}{0,0.08,0.55}
\lstset{
	tabsize=2,
	showstringspaces=false,
	basicstyle=\small\ttfamily,
	keywordstyle=[1]\color{blue},
        keywordstyle=[2]\color{blue}\bfseries,
	% Extra space before and below listings
%  	aboveskip=\smallskipamount\smallskip,
% 	belowskip=\medskipamount\medskip,
	breaklines=true,
	escapeinside={/*@}{@*/},
	% captions below listings
	captionpos=b, 
        numbers=none,
        frame=none,
}

% XQuery v1.0 (2007-01), http://www.w3.org/TR/xquery/#id-terminal-delimitation
% \lstdefinelanguage{XQuery}{
% 	sensitive=true,
%         alsodigit={-},
% 	morekeywords=[1]{ancestor,ancestor-or-self,child,descendant,descendant-or-self,following,following-sibling,parent,preceding,preceding-sibling,self},
%         otherkeywords={/,::,*,[,]},
%         morekeywords=[2]{let,for,in,return,if,then,else},
%         moredelim=[s][\normalsize\ttfamily\color{darkgreen}]{<}{>}
% }
\lstset{language=SQL}

\usepackage{stmaryrd}
\usepackage{tikz}
\usepackage{apxproof}
\usepackage{mathpartir}

 
\newcommand{\typ}[3][\Gamma]{{#1} \vdash {#2} : {#3}}
\newcommand{\form}[1][\Gamma]{\mathcal{F}[#1]}


\newcommand{\sparql}{{\textsc{sparql}}}
\newcommand{\rdf}{{\textsc{rdf}}}
\newcommand{\cc}[1]{{\textsc{#1}}}
\newcommand{\ttt}[1]{{\texttt{#1}}}
\newcommand{\rmus}{{\itshape rest-\textmu}-RA}
\newcommand{\mus}{{$\mu$-RA}}
\newcommand{\sgx}{\textsc{SparqlGX}}
\newcommand{\D}{\mathcal{D}}
\newcommand{\C}{\mathcal{C}}
\newcommand{\filt}[2][\mathfrak f]{{\sigma_{#1}\left(#2\right)}}
\newcommand{\cst}{c\rightarrow v}
\newcommand{\agg}[2][]{\Theta\left(#2,g{#1},{\C}{#1}, {\D}{#1}\right) }
\newcommand{\specialagg}[4]{\Theta\left(#4,#1,{#2}, {#3}\right) }

\newcommand*\typej[3]{#1 \vdash #2 : #3}

\newcommand{\sem}[2]{{\llbracket#1\rrbracket}_{#2}}
\newcommand{\trad}[2]{{{#1}(#2)}}
\newcommand{\dom}[1]{dom(#1)}
\newcommand{\rename}[3]{\rho_{#1}^{#2}\left(#3\right)}
\newcommand{\drop}[2]{\widetilde{\pi}_{#1}\left(#2\right)}
\newcommand{\proj}[2]{\pi_{#1}\left(#2\right)}
\newcommand{\mult}[3]{\beta_{#1}^{#2}\left(#3\right)}


\newcommand{\AJoin}{\triangleright}
\newcommand{\NJoin}{\bowtie}
\newenvironment{hproof}{%
  \renewcommand{\proofname}{Proof Sketch}\proof}{\endproof}

\newcommand{\RJoin}{\mathbin{\NJoin_{\hspace{-0.3em}R\hspace{0.3em}}}}

\newcommand{\letb}[3][X]{\mbox{let }\left(#1=#2\right)\mbox{ in }#3}
\newcommand{\fixpt}[2][X]{\mu {\left(#1=#2\right)}}

% the competitors, we cannot provide the full name of all of them
\newcommand{\postgres}{\textbf{P}}
\newcommand{\prototype}{\textbf{P}'}
\newcommand{\virtuoso}{\textbf{V}}
\newcommand{\logicblox}{\textbf{L}}
\newcommand{\neofourj}{\textbf{N}}

% I suggest we use the following macros to denote column names src and trg, no matter how we finally call them
\newcommand{\srcCol}{\texttt{src}}
\newcommand{\trgCol}{\texttt{trg}}
\newcommand{\edgeLabelCol}{\texttt{l}}

%translation functions
\newcommand*\trans[1]{\llparenthesis{#1}\rrparenthesis}
\newcommand{\muset}[1]{\left\{{#1}\right\}}  
\newcommand{\transVertexFilter}[3]{\theta_{#1}^{#2}\left(#3\right)}
\newcommand{\transProj}[2]{\Pi\left(#1\right)_{#2}}
\newcommand{\combine}[2]{\texttt{combine}\left({#1}\right)_{#2}}

%xps:
\newcommand{\yago}{\textsc{yago}}

\newcommand{\queryplan}[1]{\ensuremath{\mathcal{P}_{#1}}}



\setbeamertemplate{navigation symbols}{}
\setbeamertemplate{footline}[frame number]
%\setbeamertemplate{footline}{\null\hfill\insertframenumber}


\newcommand*\blue[1]{{\color{blue}#1}}
\newcommand*\dblue[1]{{\color{darkblue}#1}}
\newcommand*\green[1]{{\color{darkgreen}#1}}
\newcommand*\red[1]{{\color{red}#1}}
\begin{document}


\title{Translating \mus{} into SQL}
\date{3rd October 2019}

\begin{frame}
  \titlepage
\end{frame}

\begin{frame}
  \begin{itemize}
  \item Mostly straightforward

    (\mus{} is a variant of RA, and SQL is based on RA)
    \pause
  \item Need \blue{type information} to transform \mus{}’s anti-projections into SQL’s native projections.

    (\blue{Type} of a relation = the names of its attributes)

    \noindent\pause\hfill\green{This information can be inferred.}

    \pause
  \item How to translate fixpoints ?
    \begin{itemize}
      \pause
    \item We first \blue{decompose} fixpoints :
      $\fixpt\varphi = \fixpt{\kappa\cup\psi}$ where $\kappa$ does not
      contain $X$ and $\psi$ depends \blue{linearly} of $X$, i.e.
      every tuple of $\psi$ is obtained from exactly one tuple of $X$.

      \pause\dblue{Theorem:} Decomposition is always possible.

      \pause\dblue{Property:} Decomposed fixpoints can be computed
      iteratively by binding $X$ initially to $\kappa$ and then to
      only the new tuples generated each time.
\pause\item Fixpoint terms can be translated into recursive SQL
\blue{if} $\psi$ contains exactly one occurrence of $X$.

\noindent\pause\hfill\green{This is always the case for raw translated
UCRPQs.}
\pause\item When they cannot, we can use the DBMS’s procedural
language to execute a \texttt{WHILE} loop computing the fixpoint into
a temporary table.
  \end{itemize}
\end{itemize}
\end{frame}

\begin{frame}[fragile]
  \frametitle{So what exactly do we feed PostgreSQL?}
  % \begin{itemize}
  % \item A sequence of \lstinline|CREATE TEMPORARY| statements
  %   translating fixpoint terms;
  % \item A main \lstinline|SELECT| query referencing the temporary
  %   elements.
  % \end{itemize}
  % \pause
  \begin{align*}
    \trans{X} &= \text{\lstinline{SELECT * FROM X}}\\
    \trans{|c_i\to v_i|_{1\leqslant i\leqslant n}} &=
      \text{\lstinline{SELECT} } v_1 \text{ \lstinline{AS} }
       c_1,\ldots, v_n \text{ \lstinline{AS} } c_n\\
    \trans{\filt{\varphi}} &= \text{\lstinline{SELECT * FROM}
                             }(\trans{\varphi}) \text{
                             \lstinline{WHERE} } \mathfrak{f}\\
    \trans{\drop a \varphi} &= \text{\lstinline{SELECT <other cols>
                              FROM} }(\trans\varphi)\\    
    \trans{\rename a b \varphi} &= \text{\lstinline{SELECT a AS b,
                                  <other_cols> FROM}
                                  }(\trans\varphi)\\
    \trans{\mult a b \varphi} &= \text{\lstinline{SELECT a AS b
                                  <all_cols> FROM}
                               }(\trans\varphi)\\
    \trans{\varphi\cup\psi} &= \text{\lstinline{SELECT * FROM}
                              }(\trans{\varphi}) \text{
                              \lstinline!UNION SELECT * FROM !} (\trans{\psi})\\
    \trans{\varphi\NJoin\psi} &= \text{\lstinline{SELECT * FROM} }(\trans{\varphi}) \text{ \lstinline|NATURAL
                                JOIN| } (\trans{\psi})\\
    \trans{\fixpt{\kappa\cup\psi}} &= \text{\lstinline!X !}
                                     \textit{ where we add
                                     before the query the statement:}
  \end{align*}
\small
  \begin{lstlisting}[mathescape]
CREATE TEMPORARY RECURSIVE VIEW X (<type>) AS
    SELECT * FROM $\trans\kappa$
  UNION
    SELECT * FROM $\trans\psi$;    
\end{lstlisting}
\vfill\null
\end{frame}

\begin{frame}[fragile]
  Or, if $X$ appears more than once in $\psi$:
    \begin{lstlisting}[mathescape]
CREATE TEMPORARY TABLE TMP AS
 (SELECT * FROM $\trans\kappa$);
DO $\$\$$BEGIN
 CREATE TEMPORARY TABLE X AS
  (SELECT * FROM TMP);
 WHILE EXISTS (SELECT 1 FROM X) LOOP
  CREATE TEMPORARY TABLE NEW_LINES AS
   (SELECT * FROM $\trans\psi$ EXCEPT SELECT * FROM TMP);
  INSERT INTO TMP (SELECT * FROM NEW_LINES);
  DROP TABLE X;
  ALTER TABLE NEW_LINES RENAME TO X;
 END LOOP;
END;$\$\$$
DROP TABLE X;
ALTER TABLE TMP RENAME TO X;    
  \end{lstlisting}

\end{frame}

\end{document}
